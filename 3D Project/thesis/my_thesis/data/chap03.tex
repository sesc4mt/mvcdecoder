
%%% Local Variables: 
%%% mode: latex
%%% TeX-master: t
%%% End: 

\chapter{解码器性能优化方案}
\label{cha:optapproach}

\section{软硬件平台说明}
\label{sec:platformdesc}

我们的解码器支持多种硬件平台,包括x86、CELL等处理器;多种操作系统,包括Linux和Windows。本文所做的优化如无特殊说明,皆与软硬件平台无关,可以直接在其他平台上应用。为了方便实验,实验主要在以下环境下进行:

\begin{itemize}
\item {硬件平台}

\begin{itemize}
\item Intel Core2 Quad Q9400 @ 2.66GHz
	\footnote{SpeedStep功能关闭}
\item 4GB DDR2/800 Memory
\item NVIDIA GeForce GTS 250 with 512MB
\end{itemize}

\item {软件平台}

\begin{itemize}
\item Microsoft Windows XP Professional SP3
	\footnote{Lenovo ThinkCenter M6100T预装操作系统}
\item NVIDIA WHQL Driver v197.13
\item DirectX 9.0c
\item Microsoft Visual Studio 2008 SP1
	\footnote{获取自:\href{http://helpdesk.tsinghua.edu.cn/yhfw/yhfw_zbrj_tz.jsp}{清华大学校园正版软件服务}}
\item CUDA SDK 3.0
\item Intel VTune Performance Analyzer 9.1 Build 385
	\footnote{30天评估版,序列号为VVVC-BDGCWFJC}
\item Intel C++ Compiler Professional 11.1.038
	\footnote{授权同VTune}
\end{itemize}

\end{itemize}


\section{解码器性能分析}
\label{sec:decoderprofiling}

对解码器进行优化的过程中,首先需要对其现有的性能表现有一个细致的了解。对于每一个函数的被调用次数、运行时间、运行时间百分比等指标都需要有较为精确的测量,才能更好地进行接下来的优化。

对此,我使用了两个性能分析工具来分析MVC Decoder工程。分别在接下来两节中说明。

\subsection{VS2008内置性能分析}
\label{subsec:vsprofiling}
Visual Studio 2008 Team System版内置了一个Analyze功能,可以对目标程序进行性能分析。性能分析有两种:
\begin{itemize}
\item 一种是不改变编译结果,通过运行时采样,得到每个采样时刻正在运行的函数,经过汇总后的到函数在采样上的分布情况,由此估计每个函数运行时间占总运行时间的百分比。这种方式称为Sampling方式。
\item 另一种是在编译时加入一些辅助代码,运行时通过这部分代码来标志进入和推出一个函数的时间,借助这些隐藏的输出信息得到总运行时间在各个函数内部的分布。这种方式称为Instrumental Code方式。
\end{itemize}

\subsubsection{性能分析步骤}
\label{subsubsec:profilingprocess}
\subsubsection{性能分析结果}
\label{subsubsec:reportexerpt}
\subsubsection{分析结果说明}
\label{subsubsec:commentonreport}

\subsection{VTune分析}
\label{subsec:vtuneprofiling}

在实验过程中,我们发现大部分函数运行总时间没有明显差别,与我们进行项目合作的北京世纪鼎点软件有限公司的罗翰先生指出,VS2008的性能分析可能不够准确,使用Intel的VTune或许能够的到更准确的性能分析结果。

关于Intel VTune如何精确地进行性能分析,文\onlinecite{levinthal2007,levinthal2006,levinthal2008,levinthal2008a}中有介绍。

\subsubsection{性能分析步骤}
\label{subsubsec:profilingprocess}
\subsubsection{性能分析结果}
\label{subsubsec:reportexerpt}
\subsubsection{分析结果说明}
\label{subsubsec:commentonreport}

\section{优化方案}
\label{sec:optapproach}

\subsection{重写函数逻辑}
\label{subsec:rewritelogic}

\subsubsection{重写函数逻辑的例子}
\label{subsubsec:egrewritelogic}

\subsection{循环的优化}
\label{subsec:loopopt}

\subsubsection{循环优化的例子}
\label{subsubsec:egloopopt}

\subsection{汇编优化}
\label{subsec:asmopt}

\subsubsection{汇编优化的例子}
\label{subsubsec:egasmopt}

\subsection{CUDA优化}
\label{subsec:cudaopt}

\subsubsection{CUDA优化的例子}
\label{subsubsec:egcudaopt}

\section{优化目标}
\label{sec:optaim}

我们进行MVC解码器优化的目标是为了能让普通用户使用PC作为终端能够收看3D视频。在显卡尚未内置MVC硬解码器的情况下,目前所有的解码任务都交给CPU来完成。我们将优化的目标设定在用户使用主流CPU能够进行两路标清视频\footnote{关于视频分辨率,有标清\href{http://en.wikipedia.org/wiki/Standard-definition_television}{SDTV}、增强型标清\href{http://en.wikipedia.org/wiki/Enhanced-definition_television}{EDTV}和高清\href{http://en.wikipedia.org/wiki/High-definition_television}{HDTV}等制式,我们所说的标清指的是国内广泛使用的\href{http://en.wikipedia.org/wiki/Enhanced-definition_television}{EDTV}中的\href{http://en.wikipedia.org/wiki/Phase_Alternating_Line}{PAL}制式视频,分辨率为$720\times576$。}的实时解码。

量化的指标就是,用CPU进行两路分辨率为$720\times576$的视频,每一路都能达到30帧/秒,总计60fps的解码速率。

达到上述目标之后,我们的解码器就能够用来在双目3D显示平台下开展实际应用了。如果想要使用我们同是拥有的Bolod生产的裸眼观看的3D电视,则需要输出8路信号。这在目前的CPU软解码算法上较难实现,目前有两种解决方案,一种是直接利用GPU加速解码过程,软解8路信号,另一种是CPU解码出2路信号,再用GPU通过2路立体信号合成出8路需要的信号。前一种方式已经在NVIDIA的蓝光播放器中应用了,不过其解决方案并不开源;后一种两路信号合成八路信号的项目,清华大学媒体所的李化常师兄正在进行中。在2010年3月已经实现了合成算法,将两帧画面合成出八帧大约耗时1秒,目前正在进行算法的优化工作。