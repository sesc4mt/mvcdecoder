\begin{denotation}

%\item[HPC] 高性能计算 (High Performance Computing)
%\item[cluster] 集群
%\item[Itanium] 安腾
%\item[SMP] 对称多处理
%\item[API] 应用程序编程接口
%\item[PI]	聚酰亚胺
%\item[MPI]	聚酰亚胺模型化合物,N-苯基邻苯酰亚胺
%\item[PBI]	聚苯并咪唑
%\item[MPBI]	聚苯并咪唑模型化合物,N-苯基苯并咪唑
%\item[PY]	聚吡咙
%\item[PMDA-BDA]	均苯四酸二酐与联苯四胺合成的聚吡咙薄膜
%\item[$\Delta G$]  	活化自由能~(Activation Free Energy)
%\item [$\chi$] 传输系数~(Transmission Coefficient)
%\item[$E$] 能量
%\item[$m$] 质量
%\item[$c$] 光速
%\item[$P$] 概率
%\item[$T$] 时间
%\item[$v$] 速度
%\item[劝  学] 君子曰:学不可以已。青,取之于蓝,而青于蓝;冰,水为之,而寒于水。
%  木直中绳。(车柔)以为轮,其曲中规。虽有槁暴,不复挺者,(车柔)使之然也。故木
%  受绳则直, 金就砺则利,君子博学而日参省乎己,则知明而行无过矣。吾尝终日而思
%  矣,  不如须臾之所学也;吾尝(足齐)而望矣,不如登高之博见也。登高而招,臂非加
%  长也,  而见者远;  顺风而呼,  声非加疾也,而闻者彰。假舆马者,非利足也,而致
%  千里;假舟楫者,非能水也,而绝江河,  君子生非异也,善假于物也。积土成山,风雨
%  兴焉;积水成渊,蛟龙生焉;积善成德,而神明自得,圣心备焉。故不积跬步,无以至千
%  里;不积小流,无以成江海。骐骥一跃,不能十步;驽马十驾,功在不舍。锲而舍之,朽
%  木不折;  锲而不舍,金石可镂。蚓无爪牙之利,筋骨之强,上食埃土,下饮黄泉,用心
%  一也。蟹六跪而二螯,非蛇鳝之穴无可寄托者,用心躁也。\pozhehao{} 荀况
\item[CABAC] 基于上下文的自适应二进制算数编码(context-based adaptive binary arithmetic coding):CABAC的设计概念对于发生机率 > 0.5 的事件有效地编码,改进了传统霍夫曼编码法需要大量的乘法运算的问题,而在效能与压缩效率上取得相当大的改善空间。。
\item[CAVLC] 基于上下文的自适应变长编码(context-based adaptive variable-length code):适用于DCT转换后的整系数矩阵,经过zig-zag顺序扫描之后,在最高层的系数通常为+1/-1;又取得以zig-zag顺序扫描时,连续出现的0,或非零系数的总数、最后一个非零系数前零的数目等参数,作为查表时的坐标。CAVLC针对不同的块大小设计了不同的查找表,对各种不同的上下文, 使用不同的查找表进行编码,有效缩短输出比特流长度。。

\item[GOP]	group of pictures:一个GOP中所有帧的参有固定的参考结构;下一个GOP中的帧与上一个GOP中的帧的参考结构一样。

\item[MB]	宏块 (macroblock):一个16$\times$16的亮度块采样和对应的两个色度块采样。

\item[MVC]	多视点视频编解码 (multi-view video coding)。

\item[NAL unit] 网络抽象层(network abstract layer)单元:一个语法结构,包含后续数据的类型指示和所包含的字节数,数据以RBSP形式出现,必要时其中还散布有防伪字节。

\item[RBSB] 原始字节序列载荷(raw byte sequence payload):一个语法结构,包含整数个封装于NAL单元中的字节。RBSP或为空,或包含具有数字比特串形式的语法元素,其后跟随RBSP截止位和零个或多个连续的0值比特。RBSP的截止位为值为1的比特,RBSP的数据比特串的结束位置可以通过搜索RBSP中最后一个非零比特(截止位)得到。

\item[SODB] 数据比特串(string of data bits):表示语法元素的若干比特位的序列,出现在RBSP截止位之前。在SODB中,最左边的比特位是第一位并且是最高位,最右边的则是最后一位且是最低位。

\item[语法元素] syntax element:比特流中表示数据的元素。

\item[语法结构] syntax structure:零个或多个语法元素按照规定顺序一起出现在比特流中。

\end{denotation}
