
%%% Local Variables: 
%%% mode: latex
%%% TeX-master: t
%%% End: 

\chapter{解码器单核性能优化}
\label{cha:singlecoreopt}

按照第\ref{sec:optapproach}节中提到的优化方案,我们对解码器的单核性能进行了若干优化。在下文中列举其中具代表意义的部分。

\section{函数逻辑优化}
\label{sec:singlecorelogicopt}

在MVCCommonLib模块中,PBPridict.cpp文件中包含了一些宏块预测相关的函数。此前的性能分析结果表\ref{tab:vs10}中显示,其中的macroblockGetHalfPel()函数耗时较多。这是一个使用Finite Impulse Response(FIR)计算一个Luma块的half-pel的函数。具体算法参考了《H.264 and MPEG-4 video compression, video coding for next-generation multimedia》\cite{richardson2003h}第173页起的瞄述。

函数内的一部分源代码如下:
\begin{lstlisting}[caption = {macroblockGetHalfPel()函数片段(优化前)}, label = lst:macroblockGetHalfPelorig]
`\dots`
for (int i = 0; i < hw; ++i){
	for (int j = 2; j < hh - 3; ++j){
	pif->vdata_half[0][i][j] =CLIP1((pif->vdata[i][j-2]-5*pif->vdata[i][j-1]+20*pif->vdata[i][j]+20*pif->vdata[i][j+1]-5*pif->vdata[i][j+2]+pif->vdata[i][j+3] + 16) / 32);
	}
}
for (int i = 0; i < hw; ++i){
	for (int j = 2; j < hh - 3; ++j){
		tmp[i][j] =pif->vdata[i][j-2]-5*pif->vdata[i][j-1]+20*pif->vdata[i][j]+20*pif->vdata[i][j+1]-5*pif->vdata[i][j+2]+pif->vdata[i][j+3];
	}
}
`\dots`
\end{lstlisting}

检查两次循环中对变量的访问,第二次循环需要访问vdata数组,而第一次循环只修改vdata\_half数组,对vdata没有做任何改动,所以两个循环可以合并成一个。

最后形成的代码如下:

\begin{lstlisting}[caption = {macroblockGetHalfPel()函数片段(优化后)}, label = lst:macroblockGetHalfPelopt]
`\dots`
for (int i = 0; i < hw; ++i)
	for (int j = 2; j < hh - 3; ++j){
		pif->vdata_half[0][i][j] =CLIP1((pif->vdata[i][j-2]-5*pif->vdata[i][j-1]+20*pif->vdata[i][j]+20*pif->vdata[i][j+1]-5*pif->vdata[i][j+2]+pif->vdata[i][j+3] + 16) / 32);
		tmp[i][j] =pif->vdata[i][j-2]-5*pif->vdata[i][j-1]+20*pif->vdata[i][j]+20*pif->vdata[i][j+1]-5*pif->vdata[i][j+2]+pif->vdata[i][j+3];
	}
`\dots`
\end{lstlisting}

这个优化使程序运行时间\footnote{这里所说的程序运行时间是指用release版本的解码器解码一个长度为65帧,分辨率为$720\times576$,路数为2的Multi-view视频所耗费的时间。下同。}缩短了1016ms\footnote{原始的版本总运行时间为26344ms。}。

同样在MVCCommonLib模块中的PBPridict.cpp里,macroblockPredGetDataY()和macroblockPredGetDataUV()函数也是运行耗时很多的函数,事实上,不论是Visual Studio还是VTune的分析结果,这两个函数都是占用时间最多的(见表\ref{tab:vs10}和表\ref{tab:vtune10})。

这两个函数大部分的时间耗费在memset和memcpy上。对于这些系统api的调用,我们无法作修改,但是仔细观察其逻辑,我们发现函数的逻辑判断结构可以做一定优化。

源代码如下:

\begin{lstlisting}[caption = {macroblockPredGetDataY()函数片段(优化前)}, label = lst:macroblockPredGetDataYorig]
`\dots`

`\dots`
\end{lstlisting}

观察发现,

经过优化后的代码如下:

\begin{lstlisting}[caption = {macroblockPredGetDataY()函数片段(优化后)}, label = lst:macroblockPredGetDataYopt]
`\dots`

`\dots`
\end{lstlisting}

对macroblockPredGetDataUV()的优化同理。这两个函数优化后,程序运行时间再次减少了437ms。

在macroblockGetPred\_axb()中,原始的程序一共用了超过20个if语句来决定如何进行预测。根据我在《数字逻辑设计》以及《计算机组成原理》课程中的实验经验,这样的多重if语句用switch语句替换不但能使代码逻辑更加清晰,而且能提高性能。这相当于是把许多串联的2位的MUX改写成一个多位的MUX。

此处优化使程序运行时间缩短了47ms。

\section{循环优化}
\label{sec:singlecoreloopopt}

% macroblockGetPred_axb
% 循环变量的声明 1031ms
% 循环内外层对cache的优化

\section{汇编优化}
\label{sec:singlecoreasmopt}

%此处说idct8x8

\section{其他优化}
\label{sec:singlecoreothers}

% 内容相关的判断顺序微调
% 编译器ICC