
%%% Local Variables: 
%%% mode: latex
%%% TeX-master: t
%%% End: 

\cleardoublepage

\chapter{结论和展望}
\label{cha:conclusionandforesight}

\section{本文工作总结}
\label{sec:conclusion}
视频编解码的并行在近几年CPU向多核方向发展之后成为一个热门的研究领域。最近成为标准的多视点视频由于其数据量庞大、解码运算复杂,很难依靠单核的处理器达到实用的解码速度。多核并行解码多视点视频势在必行。我们以现有的多视点视频解码器为基础,希望通过一定的优化和并行处理,使普通的PC能够进行立体视频的实时解码。

%优化结果评价
总体来说,目前达到了实验初期预定的性能优化目标。通过对解码器函数级的优化,包括函数逻辑、减少循环内部的计算量、使用汇编实现部分函数,以及一个可以稳定运行的并行解码框架的实现,最终使得多视点视频的解码可以在主流PC上实现两路标清的实时解码。

%应用价值
在解码器达到性能要求之后,我们着手设计实现了一个基于NVIDIA 3D Vision套装的立体视频播放器,播放两路立体视频时能够通过3D眼镜看到立体效果。

\section{存在的问题}
\label{sec:probremained}

目前的解码器主要存在以下问题:
\begin{enumerate}
\item 多线程解码视频时的加速比不够稳定。对于部分视频会出现增加线程反而降低解码性能的情形。
\item 对MVC标准的支持尚不完整。
\item 对一些裸眼立体显示设备要求的八路视频解码性能还达不到实时。
\end{enumerate}

3D播放器主要存在以下问题:
\begin{enumerate}
\item 播放的图像序列以磁盘文件而非内存中一段数据的形式存在,将来可能成为性能瓶颈。
\item 播放时的帧率达不到流畅观看的要求,这主要是使用的一个D3D API函数调用造成的瓶颈。
\end{enumerate}


\section{未来工作}
\label{sec:futurework}

%未来继续优化方案
针对前文提出的问题,解码器在将来还需要进行以下工作:
\begin{enumerate}
\item 对于加速比反常下降的视频,我们希望实现一个智能的判断机制,在不超过线程上限的范围内,自动使用性能最优的线程数进行解码,避免性能损失。
\item 增加对MVC标准中Main Profile和Stereo High Profile的支持,增强解码器的通用性。
\item 借助GPU进行部分计算,实现两路高清视频的实时解码和八路视频的实时解码。
\end{enumerate}

3D播放器尚需要实现以下改进:
%性能提升
%接口标准
\begin{enumerate}
\item 自行实现对渲染表面的填充函数,替换D3D API调用,突破现有的性能瓶颈,保证播放的画面流畅。
\item 对播放器的输入进行修改,使其满足一个自定义的接口,方便将播放器和解码器的输出对接,做到实时解码并播放。
\end{enumerate}


