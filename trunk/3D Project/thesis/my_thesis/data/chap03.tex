
%%% Local Variables: 
%%% mode: latex
%%% TeX-master: t
%%% End: 

\chapter{解码器性能优化方案}
\label{cha:optapproach}

\section{软硬件平台说明}
\label{sec:platformdesc}

我们的解码器支持多种硬件平台,包括x86、CELL等处理器;多种操作系统,包括Linux和Windows。本文所做的优化如无特殊说明,皆与软硬件平台无关,可以直接在其他平台上应用。为了方便实验,实验主要在以下环境下进行:

\begin{itemize}
\item {硬件平台}

\begin{itemize}
\item Intel Core2 Quad Q9400 @ 2.66GHz
	\footnote{SpeedStep功能关闭}
\item 4GB DDR2/800 Memory
\item NVIDIA GeForce GTS 250 with 512MB
\end{itemize}

\item {软件平台}

\begin{itemize}
\item Microsoft Windows XP Professional SP3
	\footnote{Lenovo ThinkCenter M6100T预装操作系统}
\item NVIDIA WHQL Driver v197.13
\item DirectX 9.0c
\item Microsoft Visual Studio 2008 SP1
	\footnote{获取自:\href{http://helpdesk.tsinghua.edu.cn/yhfw/yhfw_zbrj_tz.jsp}{清华大学校园正版软件服务}}
\item CUDA SDK 3.0
\item Intel VTune Performance Analyzer 9.1 Build 385
	\footnote{30天评估版,序列号为VVVC-BDGCWFJC}
\item Intel C++ Compiler Professional 11.1.038
	\footnote{授权同VTune}
\end{itemize}

\end{itemize}


\section{解码器性能分析}
\label{sec:decoderprofiling}

对解码器进行优化的过程中,首先需要对其现有的性能表现有一个细致的了解。对于每一个函数的被调用次数、运行时间、运行时间百分比等指标都需要有较为精确的测量,才能更好地进行接下来的优化。

对此,我使用了两个性能分析工具来分析MVC Decoder工程。分别在接下来两节中说明。

\subsection{VS2008内置性能分析}
\label{subsec:vsprofiling}
Visual Studio 2008 Team System版内置了一个Analyze功能,可以对目标程序进行性能分析。性能分析有两种:
\begin{itemize}
\item 一种是不改变编译结果,通过运行时采样,得到每个采样时刻正在运行的函数,经过汇总后的到函数在采样上的分布情况,由此估计每个函数运行时间占总运行时间的百分比。这种方式称为Sampling方式。
\item 另一种是在编译时加入一些辅助代码,运行时通过这部分代码来标志进入和推出一个函数的时间,借助这些隐藏的输出信息得到总运行时间在各个函数内部的分布。这种方式称为Instrumental Code方式。
\end{itemize}

\subsubsection{性能分析步骤}
\label{subsubsec:profilingprocess}

在VS2008的菜单中有一个“Analyze”项,选择“New Performance Wizzard”打开一个性能分析向导。

选择要分析的项目、分析的方式之后就可以开始分析了。

成功执行植入了instrumental code的可执行程序之后,profiling工具收集到大量的数据。经过一段时间(在我的机器上大约两分钟)的处理,会给出一个性能分析报告。

\subsubsection{性能分析结果}
\label{subsubsec:reportexerpt}

性能分析报告中的FunctionList视图给出了各个函数(包括系统调用)占用的时间、时间的百分比以及被调用次数等等数据。我们关心的是其中的“Exclusive Time”、“Number of Calls”两列。当然,“Exclusive Time \%”也被我们选中,因为百分比相对于毫秒数让我们能更直观地了解函数占用的时间。

% Table generated by Excel2LaTeX from sheet 'MVCDecoder100611_FunctionSummar'
\begin{table}[htbp]
  \centering
  \begin{minipage}[t]{\linewidth}
  \caption{VS2008性能分析报告(前10项记录)}
  \label{tab:vs10}
	\rowcolors[]{1}{white}{gray!15}
    \begin{tabular}{lrrr}
    \addlinespace
    \toprule[1.5pt]
    \textbf{Function Name} & \textbf{Exclusive Time\footnote{不包括函数体内调用其他函数的净运行时间}} & \textbf{Exclusive Time} \% & \textbf{Number of Calls\footnote{函数被调用次数}} \\
    \midrule[1pt]
    macroblockPredGetDataUV & 341.01 & 14.41 & 2,632,478 \\
    macroblockPredGetDataY & 321.37 & 13.58 & 1,316,239 \\
    idct4x4\_c & 316.86 & 13.39 & 5,028,960 \\
    macroblockGetPred\_axb & 181.29 & 7.66  & 1,316,239 \\
    macroblockGetPred\_axb\_Bi & 133.42 & 5.64  & 571,944 \\
    macroblockGetHalfPel & 132.47 & 5.6   & 148,552 \\
    Filter & 101.92 & 4.31  & 6,215,308 \\
    addMacroblockdata & 85.36 & 3.61  & 184,171 \\
    iquant4x4\_c & 84.48 & 3.57  & 5,028,960 \\
    macroblockPBPrediction & 70.95 & 3     & 184,171 \\
    \bottomrule[1.5pt]
    \end{tabular}
  \end{minipage}
\end{table}


分析报告中仅仅MVCDecoder.exe自身的函数就有近200条记录,而大部分都占用不足$0.05\%$的时间,优化价值不大。在表\ref{tab:vs10}中我们给出了时间占用排在前10的记录。更详细的记录参见附录\ref{cha:expdatatable}中的表\ref{tab:vs50}。

\subsection{VTune分析}
\label{subsec:vtuneprofiling}

在实验过程中,我们发现大部分函数运行总时间没有明显差别,与我们进行项目合作的北京世纪鼎点软件有限公司的罗翰先生指出,VS2008的性能分析可能不够准确,使用Intel的VTune或许能够的到更准确的性能分析结果。

关于Intel VTune如何精确地进行性能分析,文\onlinecite{levinthal2007,levinthal2006,levinthal2008,levinthal2008a}中有介绍。

\subsubsection{性能分析步骤}
\label{subsubsec:profilingprocess}

打开VTune Performance Analyzer之后,我们新建一个工程,选择采集call graph数据。配置好要运行的可执行文件和执行路径就可以开始分析了。与VS2008的分析一样,经过一段时间的执行和数据处理,我们的到一个性能分析报告。

\subsubsection{性能分析结果}
\label{subsubsec:reportexerpt}

% Table generated by Excel2LaTeX from sheet 'VTune_FunctionSummary'
\begin{table}[htbp]
  \centering
  \caption{VTune性能分析报告(前10项记录)}
  \label{tab:vtune10}
  	\rowcolors[]{1}{white}{gray!15}
    \begin{tabular}{lrrr}
    \addlinespace
    \toprule[1.5pt]
    \textbf{Function Name} & \textbf{Number of Calls} & \textbf{Exclusive Time (ms)} & \textbf{\% in Function} \\
    \midrule[1pt]
    macroblockPredGetDataUV & 2632478 & 774863 & 0.61 \\
    macroblockPredGetDataY & 1316239 & 702575 & 0.58 \\
    idct4x4\_c & 5028960 & 398076 & 1 \\
    macroblockInterDecode16x16\_y & 184171 & 328834 & 0.46 \\
    macroblockGetPred\_axb & 1316239 & 311730 & 0.11 \\
    FilterMB & 210600 & 203506 & 0.21 \\
    Filter & 6215308 & 191937 & 1 \\
    iquant4x4\_c & 5028960 & 157459 & 1 \\
    macroblockInterDecode\_uv & 184171 & 154697 & 0.48 \\
    macroblockPBPrediction & 184171 & 150609 & 0.05 \\
    \bottomrule[1.5pt]
    \end{tabular}
\end{table}


性能分析报告包含一个函数调用关系图以及函数运行时间表,我们关注的是后者。表中大部分列与VS2008的分析结果是公共的,比如函数名、总时间、净时间等等。我们采集了与此前同样意义的几列数据,包括“Function”,“Calls”,“Self Time”和“\% in function”。我们在表\ref{tab:vtune10}中给出前10条记录。

\subsection{性能分析结果说明}
\label{subsec:commentonreport}

我们观察两个分析结果表\ref{tab:vs10}和表\ref{tab:vtune10}可以发现:虽然两个性能分析工具在结果上存在一些差别,但函数运行时间的排序大体上是一致的,耗时多的函数都排在表格靠上的位置,表中列出的前10位的函数有7个在两个表中都出现了。我们认为,可以根据性能分析的结果来确定函数优化的大体方向,优化的主要对象就是表中排在前列的函数。

对这些函数进行进一步观察,我们可以将它们分类:
\begin{description}
\item[简单函数多次调用] 典型的例子是idct4x4\_c和iquant4x4\_c。这两个函数的调用次数都在5028960次,这仅仅是解码65帧两路视频所需要的次数。对于这类函数,哪怕是性能上的一点点提升,都会因为巨大的调用次数而对总解码时间产生较大的影响。
\item[复杂函数] 典型的例子是macroblockGetHalfPel,这个函数仅仅执行148552次,单次执行时间是idct4x4c的14倍,iquant4x4c的53倍。这样的函数需要有很大幅度的优化才会在总时间中有所体现。
\end{description}

同时,我咨询了对整个系统十分熟悉的胡伟栋,他指出:耗时排在前列的函数大多是根据JMVC参考软件的逻辑来编写的,考虑了一些工程上扩展的需要而非性能优先,有一定的提升空间。

\section{优化方案}
\label{sec:optapproach}

在经过几次组会讨论之后,我们确定了如下的几个优化方案,分别对不同类型的函数进行重写逻辑、优化循环、汇编优化和CUDA优化。从重写逻辑开始,一步步优化解码器性能,直到符合优化目标(见\ref{sec:optaim})为止。

\subsection{重写函数逻辑}
\label{subsec:rewritelogic}

解码器中有一些函数的逻辑判断有多处重复,降低了一定的性能。如果能够理清楚函数逻辑,修改函数内部执行的顺序,就能够加快一些速度。

\subsubsection{重写函数逻辑的例子}
\label{subsubsec:egrewritelogic}

\subsection{循环的优化}
\label{subsec:loopopt}

对帧和宏块的处理常常有两层for循环便利一个width$\times$height的像素矩阵,逐像素进行操作。对于这样的函数,首先考虑是否能够将多次循环内的操作合并到少量循环,再考虑循环结构对cache的友好程度。

\subsubsection{循环优化的例子}
\label{subsubsec:egloopopt}

\subsection{汇编优化}
\label{subsec:asmopt}

在项目讨论中,罗翰先生提出ffmpeg等开源的视频项目中可能会有一些变换的汇编优化版本,如果能够应用汇编优化,那么这些函数的性能将会有大幅度的提升。就此,我们考虑对一些函数进行汇编优化。

\subsubsection{汇编优化的例子}
\label{subsubsec:egasmopt}

\subsection{CUDA优化}
\label{subsec:cudaopt}

视频处理中很常见的就是并行性无处不在,解码器的调度器保证了帧解码具有一定的并行性,而对帧和宏块内部的处理往往还有可以并行的for循环逐像素操作,利用GPU的大量核心进行简单操作可以加快这类操作的速度。

\subsubsection{CUDA优化的例子}
\label{subsubsec:egcudaopt}

\section{优化目标}
\label{sec:optaim}

我们进行MVC解码器优化的目标是为了能让普通用户使用PC作为终端能够收看3D视频。在显卡尚未内置MVC硬解码器的情况下,目前所有的解码任务都交给CPU来完成。我们将优化的目标设定在用户使用主流CPU能够进行两路标清视频\footnote{关于视频分辨率,有标清\href{http://en.wikipedia.org/wiki/Standard-definition_television}{SDTV}、增强型标清\href{http://en.wikipedia.org/wiki/Enhanced-definition_television}{EDTV}和高清\href{http://en.wikipedia.org/wiki/High-definition_television}{HDTV}等制式,我们所说的标清指的是国内广泛使用的\href{http://en.wikipedia.org/wiki/Enhanced-definition_television}{EDTV}中的\href{http://en.wikipedia.org/wiki/Phase_Alternating_Line}{PAL}制式视频,分辨率为$720\times576$。}的实时解码。

量化的指标就是,用CPU进行两路分辨率为$720\times576$的视频,每一路都能达到30帧/秒,总计60fps的解码速率。

达到上述目标之后,我们的解码器就能够用来在双目3D显示平台下开展实际应用了。如果想要使用我们同是拥有的Bolod生产的裸眼观看的3D电视,则需要输出8路信号。这在目前的CPU软解码算法上较难实现,目前有两种解决方案,一种是直接利用GPU加速解码过程,软解8路信号,另一种是CPU解码出2路信号,再用GPU通过2路立体信号合成出8路需要的信号。前一种方式已经在NVIDIA的蓝光播放器中应用了,不过其解决方案并不开源;后一种两路信号合成八路信号的项目,清华大学媒体所的李化常师兄正在进行中。在2010年3月已经实现了合成算法,将两帧画面合成出八帧大约耗时1秒,目前正在进行算法的优化工作。